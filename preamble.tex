% ------------------------------------------------------
% CSUN Preamble (preamble.tex)
% ------------------------------------------------------

% -----------------------
% URL and Hyperlink Setup
% -----------------------

\PassOptionsToPackage{bookmarks=true,
                     colorlinks=true,
                     linkcolor=black,    % internal links (ToC, \ref, \autoref) stay black
                     citecolor=black,    % citations stay black
                     urlcolor=cyan,      % pure URLs, DOIs, etc. become cyan
                     pdfusetitle}{hyperref}
\PassOptionsToPackage{hyphens}{url}
\usepackage{hyperref}

% -----------------------
% Fonts and Math Packages
% -----------------------
\usepackage{mathptmx}       % Times font
\usepackage{lmodern}        % Latin Modern fonts
\usepackage{amssymb}        % math symbols
\usepackage{amsmath}        % math environments
\usepackage{amsthm}         % theorem environments
\usepackage{latexsym}       % additional symbols

% -----------------------
% Tables and Arrays
% -----------------------
\usepackage{array}          % extended array/tabular
\usepackage{booktabs}       % professional tables
\usepackage{longtable}      % tables spanning pages
\usepackage{multirow}       % multirow cells
\usepackage{makecell}       % better cell formatting
\usepackage{adjustbox}      % for resizing tables or graphics
\usepackage{calc}           % calculations with lengths
\usepackage{setspace}
\usepackage{tabularx}
\setlength{\parindent}{2em}

% -----------------------
% Figures and Graphics
% -----------------------
\usepackage{graphicx}       % include graphics
\usepackage{pdfpages}       % include pdf pages
\usepackage{tikz}           % drawing package
\usepackage{tikz-cd}        % commutative diagrams
\usetikzlibrary{matrix, arrows.meta, decorations.pathmorphing}

% -----------------------
% Color and Highlighting
% -----------------------
\usepackage{xcolor}         % color management
\colorlet{shadecolor}{cyan!10}  % make Shaded environments a very light cyan


% -----------------------
% Code Highlighting
% -----------------------
\usepackage{fancyvrb}       % enhanced verbatim environments
\usepackage{framed}         % framed and shaded environments
% Define the Shaded environment used by knitr/Pandoc for code chunks
\newenvironment{Shaded}{\begin{shaded}}{\end{shaded}}

% Pandoc/knitr highlighting macros: include them here or input from a file
% Option 1: paste macros here (see below)
% Option 2: input from external file 'highlighting.tex'
% \input{highlighting.tex} 

% Here is a minimal example of required highlighting macros for tango style:
\makeatletter
\newcommand\KeywordTok[1]{\textcolor[rgb]{0.00,0.44,0.13}{\textbf{#1}}}
\newcommand\DataTypeTok[1]{\textcolor[rgb]{0.56,0.13,0.00}{#1}}
\newcommand\DecValTok[1]{\textcolor[rgb]{0.25,0.63,0.44}{#1}}
\newcommand\BaseNTok[1]{\textcolor[rgb]{0.25,0.63,0.44}{#1}}
\newcommand\FloatTok[1]{\textcolor[rgb]{0.25,0.63,0.44}{#1}}
\newcommand\CharTok[1]{\textcolor[rgb]{0.25,0.44,0.63}{#1}}
\newcommand\StringTok[1]{\textcolor[rgb]{0.25,0.44,0.63}{#1}}
\newcommand\CommentTok[1]{\textcolor[rgb]{0.38,0.63,0.69}{\textit{#1}}}
\newcommand\OtherTok[1]{\textcolor[rgb]{0.00,0.44,0.13}{#1}}
\newcommand\AlertTok[1]{\textcolor[rgb]{1.00,0.00,0.00}{\textbf{#1}}}
\newcommand\FunctionTok[1]{\textcolor[rgb]{0.02,0.16,0.49}{#1}}
\newcommand\RegionMarkerTok[1]{#1}
\newcommand\ErrorTok[1]{\textcolor[rgb]{1.00,0.00,0.00}{\textbf{#1}}}
\newcommand\NormalTok[1]{#1}
\makeatother

% -----------------------
% Layout and Formatting
% -----------------------
\usepackage{fancyhdr}       % headers and footers
\usepackage{setspace}       % line spacing
\usepackage{verbatim}       % comment environment
\usepackage{lettrine}       % dropped capitals
\usepackage{titling}        % title customization
\usepackage{float}          % float control
\floatplacement{figure}{H}  % force figures here
\usepackage{rotating}       % rotate figures or tables

% -----------------------
% Chemistry
% -----------------------
\usepackage{chemfig}        % chemistry diagrams/arrows
\usepackage[version=4]{mhchem}

% -----------------------
% Miscellaneous
% -----------------------
\usepackage{ragged2e}
\usepackage{caption} % Make sure this is in your preamble
\captionsetup{width=\textwidth} % Add this once globally in preamble, or locally as below

